\chapter{Innovation ecosystems}
\label{ch:innovationecosystems}

The systems approach has been used to describe the multifaceted nature of innovation at various levels – national, regional, technological, and sectors – and to describe the processes by which research capabilities build knowledge, then transfer the knowledge to support business development in the context of the Triple Helix of business, government and academic interaction \citep{Etzkowitz2000}. The systems approach recognizes the interaction among the many actors and other “determinants of innovation processes [...] that influence the development and diffusion of innovations” \citep{Russell1999}. 

The ecosystem metaphor enriches the systems model with value and culture. A dynamic innovation ecosystem is characterized by a continual realignment of synergistic relationships that promote growth of the system \citep{Russell2015}.

The concept of an ecosystem has its roots in biology. 

\todo{Borrowed from the Edutech in Finland book's chapter (Niemi, 2015): According to the Collins English Dictionary, an ecosystem is ”a system involving the interactions between a community of living organisms in a particular area and its nonliving environment”. The concept has its roots in biology, where typical ecosystems are a forest, a pond, and grassland. The most important feature of the ecosystem is interconnectedness. Species work in close interaction to provide the necessary ingredients for their survival. Warmth, water, and energy sources all make their own contributions to the ecosystem. The concept of ecosystem has recently expanded to more general meanings, especially social structures. The systems of human actors or companies and organizations can also be described as ecosystems. In the Collins English Dictionary, an ecosystem can also be “any system of interconnecting and interacting parts”.}

\todo{Figure out a street-credible way to refer to the general meaning of a concept. Dictionary, Wikipedia, Wiktionary, 101 textbook?}

Understanding open innovation, value co-creation, business ecosystems and innovation ecosystems all insist on investigating activities that take place rather in between than within organizations. Interdependency and interconnectedness are the two key features of ecosystems. The ecosystem concept is applied extensively in different socio-technical fields and discourses from technology ecosystems to Web 2.0 and digital ecosystems \citep{Dini2005}. Virtues such as interoperability, integration and collaboration as well as technology-centric constructs such as APIs (Application Programming Interface) and mashups are often associated with the ecosystem concept.

Particularly business ecosystems have disrupted existing ways organize activities related to business and innovation. \todo{"Uusi tapa organisoida työn- ja tulonjako ekosysteemissä syrjäytti aikaisemmin keskeisen toimijan", kuvailee Järvi.} \cite{Moore1993} introduced the concept into the context of business strategy in his seminal article on new ways for companies to allow other companies create value for the focal company. Business ecosystems are composed of companies that co-evolve, co-create value and are co-dependent. A business ecosystems often operate on top of a technology platform that provides a shared infrastructure and a set of technology assets to develop new applications and services -- i.e. to create value -- and means for customers to use the developed services and applications in a consistent manner and, importantly, in a way that companies creating the value are also able to capture or appropriate the value following their business models.  

Companies take – and evolve into – different roles in the ecosystem. Focal companies are the ones that build new platforms for others to operate. These other are often called complementors. Some companies are boundary crossers, operating in a number of individual platforms. Others are boundary-spanners who effectively build new platforms on top of the existing ones. \citep{Jarvi2013}\todo{Add a reference to the Hanken book (Tukiainen, Burström and Lindell, 2015) or to \cite{Burstrom2014} if you manage to get a copy.} 

Innovation ecosystem is a relatively new and therefore more ambiguous concept compared to business ecosystem. It is being used in different ways in literature and practical applications. \cite{Hwang2012}, for example, have integrated the ecosystem metaphor in their rainforest framework for "building the next Silicon Valley". To make a distinction between ecosystems of business and innovation in the context of this dissertation, we point to their expected outputs. When the key artifacts created in business ecosystems are products and services, we see that the main output of innovation ecosystems are new companies, the increase of information flow through pre-competitive collaboration, and business-relevant new knowledge. New technologies, ideas, questions. Innovation ecosystems survive through a constant idea flow, re-configuration, and evolution.

In this dissertation, we investigate innovation ecosystems with a Silicon Valley-style mindset. We study the structures that emerge from activities taking place around startups, the individuals that found, advice, invest into, and work in key positions for the startups. Also growth companies that have already crossed the death valley of company growth and continue to evolve toward a liquidity event, e.g. exit through an acquisition or Initial Public Offering (IPO) are of our interest. For context, we also study the already established enterprises and deals and alliances connecting them with each other. Moreover, we seek ways to explore the role of individual customers as well as organizations that facilitate the creation and growth of the companies in the core of our investigations. 

% - - - -

% Today, we talk about innovation ecosystem in the context of learning and learning technology.

% When we were preparing to give a talk about the topic last year in conjunction w/ a conference on ... Here at Stanford, to me - a Finn that has studied 9 + 3 + 10 + 8 years and also studied the topic in the beginning of 2000 millennium, the results were mind-blowing. The abundance of activity, the dense networks of companies, individual people bursting off from universities to start up their companies and scaling them up with the help of venture capital was - to say the least - different to what things look like back home. Both in general as well as particularly in terms of educational technology specifically.

% We see that a relatively small number of universities are in a key role in the venture capital-backed innovation ecosystem.

% Moreover, some individuals have a special role in building bridges between the different ecological niche.

% We are also able to take a domain-level view on the sectors, technologies, markets etc. the companies in learning technology are also targeting/addressing.

\section{Analyzing innovation ecosystems as networks}

Innovation ecosystems are a complex phenomenon, with multiple entities connected through multiple level relationships, as well as multiple stakeholder perspectives into those relationships. Ecosystems that promote innovation have become a quest for companies, cities, regions and countries. It is agreed that “relationships shape the behavior and outcome of all stakeholders as well as the system-level effects” \citep{Hwang2012}, and that it is through the relationships of individuals within and across organizations in an ecosystem that knowledge transfer, technology dissemination and organizational change are accomplished \citep{Russell2015}. Program managers and policy analysts in charge of transforming innovation ecosystems seek to define and describe innovation ecosystems in order to set goals, determine interventions and evaluate change, and visualizing the innovation ecosystem has proven instrumental to strategy setting and decision-making \citep{Still2014}. By making the roles and relationships explicit, both quantitative analyses and visualizations can be used to support the creation and management of innovation ecosystems. By tracking the provenance of data and authorship of analytical refinements, the collaborative exploration gains transparency.