\chapter{Network analytics}
\label{ch:networkanalytics}

\section{Background}

Existing research on networks shows that network analysis has a good fit for explorative analysis of innovation ecosystems: much is already known about structure in networks \citep{Barabasi2003,Granovetter1973,Watts1998,Watts1999}, the roles of individual actors in a network \citep{Hansen2011}, the drivers of network evolution \citep{Giuliani2008} as well as the latent structures and dynamics behind the diffusion of information through networks \citep{Leskovec2009}, network control \citep{Liu2011} and virality \citep{Shakarian2013,Weng2013}. \cite{Moreno1953,Freeman2000,Hansen2009,Hansen2011,Russell2011,Still2014,Basole2012,Ritala2011} and \cite{Ritala2014} give additional examples on using a network approach to investigate complex phenomena that are driven by sets of interconnected actors.

\section{Visual network analytics}

Innovation ecosystems are studied in this dissertation with visual network analytics. Network analysis has its roots in Social Network Analysis (SNA) \citep{Moreno1953,Wasserman1994}. With visual network analytics, we refer to taking a visual analytics \citep{Thomas2006,Heer2012} approach to network analysis. Visual analytics stresses the process-centric, interactive nature of using visualizations in supporting data-driven investigations \citep{Keim2010,Heer2012}. Visual analytics is particularly suitable approach for exploring new phenomenon with a data-driven approach. Visual network analytics allows the emergence of insights on the structure and dynamics of innovation ecosystems, social media platforms and other networked phenomena. Transforming those insights into action requires communicating the insights to constituents of change \citep{Russell2011,Still2014}. Visual network analysis is a valuable method for investigating social configurations and for interactively communicating their findings to others \citep[][]{Freeman2009}.

A data-driven process for understanding the roles of the different actors in the innovation ecosystem allows for interactive discovery to support orchestration of innovation ecosystems \citep{Russell2015}. Multiple perspectives can be invited and exchanged in the process of developing and orchestrating transformation programs. With subsequent automation of data updates and tracking analyses, the assumptions and contingencies underlying decisions can be monitored for changes that would impact policy and program directions.

In the case studies included in this dissertation described in detail in Chapter~\ref{chapter:caseresults}, we use network nodes to represent the innovation ecosystem actors. The actor are connected to each other on basis of different kinds of transactions between them. These transactions include investments, acquisitions, deals and alliances. Moreover, key individuals are connected to companies they are or have been affiliated with. Network metrics are used to quantify the structural positions of these actors. Network representations are laid out in a way that allows their visual investigation. Visual properties of network nodes and edges are specified in a way that supports these investigations.

\section{Cognitive fit of network analysis}

To kick of the rebuilding process of the world after the Second World War, Vannevar \cite{Bush1945} introduced a fictional device Memex and, as part of the description of the device, was the first to present the idea of hyperlink. The leading idea was to connect pieces of information by association (instead of indexing), so in a similar fashion than the human mind operates \citep{Bush1945}: 

\begin{quote}
The human mind [...] operates by association. With one item in its grasp, it snaps instantly to the next that is suggested by the association of thoughts, in accordance with some intricate web of trails carried by the cells of the brain. It has other characteristics, of course; trails that are not frequently followed are prone to fade, items are not fully permanent, memory is transitory. Yet the speed of action, the intricacy of trails, the detail of mental pictures, is awe-inspiring beyond all else in nature.” 
\end{quote}

We claim that representing innovation ecosystems as networks allows for an intuitive way to investigate and revisit data representing innovation ecosystems. Indeed, \citep{Bush1945} added that “[m]an cannot hope fully to duplicate this mental process [the intricacy of trails in human mind] artificially, but he certainly ought to be able to learn from it”. In this dissertation, we taken advantage of existing sources of digital data on innovation ecosystem actors and transactions and affiliations between them to recreate these traces to allow for system-level views of the currently scattered innovation ecosystem landscape. To reuse McGonigal’s \citeyearpar{McGonigal2005} inspirational metaphor inspired by Lewis Carroll, we see visual network analytics as meas to craft “rabbit holes” through which innovation ecosystem explorers are drawn into new information landscapes beyond their imagination to find new, interesting ideas, companies, investors, communities or a similar-minded innovation champions \citep[cf.][]{Huhtamaki2007}.

The well-known mantra of visual information retrieval iterates over three phases: "Overview first, zoom and filter, then details-on-demand" \citep{Shneiderman1996}.  
In the context of visual network analysis, however, users may prefer to follow the process of “start with what you know, then grow” \citep[][p. 35]{Heer2005}. Visual analytics theory suggest that, at best, an investigator of a phenomena is able to interact with all the phases of the process from view creation to exploration and refinement in an expressive manner \citep{Heer2012}.

Cognitive fit theory suggests that in supporting problem-solving and decision-making, it is particularly important to find a fit between the problem-solving task and the problem representation and supporting tools; cognitive fit allows for faster and more accurate performance in decision-making \citep{Vessey1991}. 

There is a delicate balance in developing tools with cognitive fit, however: experimental research shows that neither the time used to conduct a task nor the confidence that the users feel about their decisions are good predictors of the accuracy of their insights or decisions \citep[cf.][]{Dunn2001}. 

%From \cite{Basole2015EnablingWideLens}: "Cognitive fit theory further suggests that when the problem representation fits the problem-solving task, a preferable and more consistent mental representation of the problem will be realized, thereby facilitating the problem solving process, and consequently resulting in preference for the representation, along with faster and more accurate performance in decision-making."}

Using network analysis in the context of measuring and orchestrating innovation ecosystems is closely related to relational and social capital theory. In their seminal article on social capital\footnote{\cite{Nahapiet1998} is a key article in between social capital and relational capital research \citep[cf.][]{Still2013relationalsocialcapital}.}, \cite{Nahapiet1998} introduce a third dimension for social capital to complement the structural dimension (system-level social capital):

\begin{quote}
The third dimension of social capital, which we label the "cognitive dimension," refers to those resources providing shared representations, interpretations, and systems of meaning among parties (Cicourel, 1973). We have identified this cluster separately because we believe it represents an important set of assets not yet discussed in the mainstream literature on social capital but the significance of which is receiving substantial attention in the strategy domain (Conner \& Prahalad, 1996; Grant, 1996; Kogut \& Zander, 1992, 1996). 
\end{quote}

From relational capital research viewpoint, this dissertation seeks to use visual network analytics to augment the cognitive dimension of ecosystemic relational capital \citep{Still2014EcosystemicRelationalCapital}.
