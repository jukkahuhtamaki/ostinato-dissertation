\chapter{Tiivistelmä}

\textbf{Mitä?} Innovaatiotoiminta siirtyy enenevässä määrin organisaatioiden ulkopuolelle ja organisaatioiden väliin. Avoin innovaatio, yhteiskehittely, käyttäjälähtöisyys, API- ja alustatalous ja liiketoimintaekosysteemit ovat muutoksen keskeisiä ajureita. Liiketoimintaekosysteemeistä kumpuava innovaatioekosysteemien teoria määrittelee uuden viitekehyksen innovaatiotoiminnan analyysille ja tutkimukselle ja siten innovaatioekosysteemien mittaamiselle. Innovaatioekosysteemit ovat kononaisvaltaisia organisaatioiden rajat ylittäviä, poliittisia, taloudellisia ja teknologisia innovaatiojärjestelmiä, jotka tarjoavat viljavat puitteet liiketoiminnan kasvun kiihdyttämiselle ja jatkuvuudelle kestävällä tavalla.

Tässä toimintatutkimuksen ja design-tutkimuksen otteilla toteutetussa väitöskirjassa kehitetään kokonaisvaltainen prosessimalli innovaatioekosysteemien datalähtöisen laskennallisen visuaalisen rakenteellisen analyysin tueksi. Verkostoanalyysi tarjoaa luontevan keinon innovaatioekosysteemien rakenteellisen analyysin toteuttamiseen. Datalähtöisyys viittaa tässä työssä kykyyn kerätä dataa erilaisista digitaalisista lähteistä analyysin lähteeksi ja analysoida dataa automatisoidusti innovaatioekosysteemien tutkimuksen tueksi.

\textbf{Miksi?} Innovaatioekosysteemien järjestemätason visuaalinen rakenneanalyysi tukee kolmea keskeistä innovaatioekosysteemeihin liittyvää toimintaa: innovaatioekosysteemien analytiikkaa päätöksenteon tukena, innovaatioekosysteemien akateemista tutkimusta sekä sekä innovaatioekosysteemien johtamista orkestroinnin periaatteiden mukaisesti. 

\textbf{Miten?} Innovaatioekosysteemin toimijat esitetään solmuina, jotka kytketään toisiinsa toimijoiden välisten yhteyksien ja transaktioiden perusteella. Verkostot visualisoidaan eksploratiivisen analyysin ja tulosten raportoimisen tueksi. Verkostoanalyysin tunnusluvut mahdollistavat erilaisten toimija- ja järjestelmätason ominaisuuksien esittämisen tunnuslukuina. Näitä tunnuslukuja voidaan käyttää sekä innovaatioekosysteemien määrällisessä tutkimuksessa että visualisointien ominaisuuksien määrittelemisessä

\textbf{Tulokset ja kontribuutio?} Väitöskirjatutkimuksen keskeinen tulos on Ostinato-malli. Ostinato-malli jakaa innovaatioekosysteemin analyysiprosessissa kahteen vaiheeseen, 1) datan kerääminen ja jalostaminen sekä 2) verkoston luominen ja analyysi. Datan kerääminen ja jalostaminen toteutetaan neljässä vaiheessa: entiteetti-indeksin luominen, Webin ja ohjelmointirajapintojen ryömiminen, datan raapiminen, ja datan koostaminen. Verkoston luominen ja analyysi muodostuu seitsemästä vaiheesta: entiteettien valinta, solmujen ja yhteyksien luominen, tunnuslukujen laskenta, solmujen ja yhteyksien suodattaminen, entiteetti-indeksi täsmentäminen ja visuaalisten ominaisuuksien määrittely. 

Ostinato-mallin ohella tässä väitöskirjassa määritellään joukko ohjenuoria innovaatioekosysteemien verkostomallinnuksen ja visuaalisen verkostoanalyysityön tueksi. Väitöskirja antaa myös oman panoksensa innovaatioekosysteemien empiiriseen tietämykseen mallintamalla joukon eri abstraktiotasolle sijoittuvia innovaatioekosysteemejä.

\chapter{Abstract}
\label{chapter:abstract}

\textbf{What?} 
% Innovation is imperative in sustaining and developing any business. Moreover, successful and sustainable businesses are key in developing thriving programs, regions, countries and continents. 
More often than ever before, innovation is taking place in between the formal organizational structures. It is taking place in innovation ecosystems. Open innovation, co-creation, user driven innovation, API and platform economy and business ecosystems are key drivers of the transformation. Innovation ecosystem theory sets a new framework for analyzing and investigating and therefore measuring the ecosystems. Innovation ecosystems are open dynamic systems that cross the organizational and geographical boundaries and include financial, technological and political dimensions. Innovation ecosystems introduce new means to develop new businesses and industries in a sustainable manner. 

\textbf{Why?} Innovation ecosystems present a natural habitat for exercising data-driven visual network analytics. Multiple sources of data with varying quality exists for tracking down activities of innovation ecosystem actors representing different organizations. Moreover, there is an imperative need for system-level view for the ecosystems. Innovation ecosystems are rather orchestrated than managed, therefore groups of people have to come together to form shared vision on a future that they want to pursue together.  

Addressing innovation ecosystems as networks allows scholars and practitioners to study their complexity, providing a means for mapping, monitoring and managing the ecosystem components. To do this, we take a data-driven network analysis approach to study innovation ecosystems in regional, metropolitan, national and international level as well as e.g. in the context of programmatic activities supporting innovation and growth. We Action Design Research approach that is based on iteration through construction of network visualizations as artifacts. We use a number of different datasets in these studies, including social media, socially constructed data available online, and proprietary sets of data represented as spreadsheets and other formats.

\textbf{How?}
Network analysis is valuable method for investigating and mapping the social structure driving all kinds of phenomena and sharing the findings with others. The interactive visual analytics approach transforms data into views that allow the visual exploration of the structure and evolution of networks represented by data, therefore increasing the transparency of the roles of innovation ecosystems actors and the patterns emerging from their individual activities. 

\textbf{Results and contribution} This dissertation makes several contributions in pushing forward the art and science of data-driven visual network analytics of innovation ecosystems. Most importantly, the dissertation presents the Ostinato Model, an iterative, user-centric, process-automated model for data-driven visual network analytics. The Ostinato Model simultaneously supports the automation of the process and enables interactive and transparent exploration. The model has two phases, Data Collection and Refinement and Network Creation and Analysis. The Data Collection and Refinement phase is further divided into Entity Index Creation, Web/API Crawling, Scraping, and Data Aggregation. The Network Construction and Analysis phase is composed of Filtering in Entities, Node and Edge Creation, Metrics Calculation, Node and Edge Filtering, Entity Index Refinement, Layout Processing and Visual Properties Configuration. A cycle of exploration and automation characterizes the model and is embedded in each phase.

In addition to the Ostinato Model, this dissertation contributes to the empirical body of knowledge on innovation ecosystems with a set of design guidelines for modeling and visualizing innovation ecosystems as networks. Finally, the dissertation contributes with a set of empirical investigations of innovation ecosystems of different levels of abstraction and complexity from platform, industry, program, national to international. Policy makers, orchestrators and other stakeholder of the innovation ecosystem under investigation have subscribed to the approach presented in this dissertation. More research and development of supporting processes and tools are needed to take full advantage of the presented approach in facilitating and orchestrating inter-organizational innovation activities. 