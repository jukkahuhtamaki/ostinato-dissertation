\chapter{Defining innovation ecosystems, network analysis and visual analytics}
\label{ch:background}

This dissertation operates across three key domains,  innovation ecosystems, network analysis, and visual network analytics. In this chapter, we review the body of knowledge in these domains from the viewpoint of the dissertation.

\section{Innovation ecosystems}
\label{sec:innovationecosystems}

The ecosystem concept has its roots in biology. According to \cite{Moran1990ThePractice}, ``ecosystem generally refers to the structural and functional interrelationships among living organism and the physical environment within which they exist.'' \cite{Moore1993PredatorsCompetition} first introduced the concept to business literature in his seminal article on business ecosystems. The article discusses new ways for companies to allow other companies to create value for them, i.e. the focal company of a business ecosystem. Moore notes that business ecosystems ``condense out of the original swirl of capital, customer interest, and talent generated by a new innovation, just as successful species spring from the natural resources of sunlight, water, and soil nutrients'' \citep{Moore1993PredatorsCompetition}. In business and innovation literature, ecosystem is used both as a metaphor \citep[e.g.][]{Hwang2012,Huhtamaki2011AFinancing}, a business strategy artifact \citep{Moore1993PredatorsCompetition}, as well as to refer to system-level analysis \citep[cf.][]{Pentland2015}.

Innovation ecosystem is a relatively new concept, even when compared to business ecosystem. It is being used in different ways in literature and practical applications. \cite{Hwang2012}, for example, have integrated the ecosystem metaphor in their rainforest framework for ``building the next Silicon Valley.'' To make a distinction between ecosystems of business and innovation in the context of this dissertation, we point to their expected outputs. When the key objective in business ecosystems is to organize value creation and value appropriation in an ecosystem setting, we see that the main output of innovation ecosystems is the increase of information flow and collaboration and therefore the creation of new business-relevant knowledge, ideas and technologies that lead to new products, successful companies and economic growth. In Moore's (\citeyear{Moore1993PredatorsCompetition}) words, innovation ecosystem is the ``swirl'' and its upstream. Innovation ecosystems survive through a constant idea flow, re-configuration, and evolution \citep[cf.][]{Pentland2015}.

Key characteristics of ecosystems are 
interconnectedness, 
interdependency,  
co-evolution, 
value co-creation, and 
co-opetition \citep{Jarvi2016TakingReview,Huhtamaki2011AFinancing}.
Actors in business ecosystems and innovation ecosystems are ``loosely interconnected'' \cite{Iansiti2004TheSustainability}.\footnote{\cite{Iansiti2004TheSustainability} stress that ``like their biological counterparts, business ecosystems are characterized by a large number of loosely interconnected participants who depend on each other for their mutual effectiveness and survival.''} Success of a given innovation often relies on the success of focal companies' environment, i.e. companies are interdependent with each other \citep{Adner2010ValueGenerations}. \cite{Thomas2012ModelingLiteratures} state that co-evolving ecosystem actors ``develop over time sympathetically with the other participants in order to maintain stability and health of the ecosystem in the face of change.''
\cite{Ramaswamy2009LeadingValue} claims that value co-creation is an emerging business and innovation paradigm that leads to the need of ``changing the very nature of engagement and relationship between the institution of management and its employees, and between them and co-creators of value - customers, stakeholders, partners and other employees.'' Finally,
\cite{Ritala2009WhatsCoopetition} note that co-opetition, i.e. collaboration with competitors, is in some cases ``an effective way of creating both incremental and radical innovations, especially in high-tech industries.'' 

\cite{Russell2011TransformingOrchestration} take an even broader scope and define innovation ecosystem as an ``inter-organizational, political, economic, environmental and technological systems of innovation through which a milieu conducive to business growth is catalyzed, sustained and supported.'' They note that in ecosystem level, individual relationships take the shape of a network ``through which information and talent flow through systems of sustained value co-creation.'' Importantly, \cite{Russell2011TransformingOrchestration} include organizational investors and invidual people--founders, advisors, and business angels--as innovation ecosystem actors (and therefore potential units of analysis in innovation ecosystem investigations) \cite[cf.][]{Huhtamaki2011AFinancing}.

To manage innovation ecosystems as well as to facilitate their emergence, the process of network orchestration is encouraged \citep{Russell2015RelationalEcosystems}. In all, innovation ecosystems are rather orchestrated than controlled or managed \citep{Russell2011TransformingOrchestration,Ritala2009,Ritala2013,Paquin2013}. When collective gains are sought at the network level, change agents seek to facilitate the emergence of networks, orchestrate existing networks and manage their growth \citep{Russell2011TransformingOrchestration}. A dynamic innovation ecosystem is characterized by a continual realignment of synergistic relationships that promote growth of the system \citep{Russell2015RelationalEcosystems}.

Innovation ecosystem research can take place in three different levels \citep{Jarvi2016TakingReview}: ``the individual actor, the relationship between actors and the ecosystem as a whole.'' We claim that system-level view is imperative in investigating, navigating, and orchestrating innovation ecosystems. With \cite{Jarvi2016TakingReview} as evidence, we note that during the writing of this dissertation, empirical ecosystem-level research on business and innovation ecosystems is scarce, very likely due to the mismatch between knowledge demand and the the existing methods that are perhaps better suited to conduct analysis in actor and relationship level.

Discussion on the utility of the innovation ecosystem as a concept is ongoing. A recent critique of the concept \citep{OhInnovationExamination} states that the eco-prefix in ecosystem may only add to the difficulties in communication related to research and its application in decision-making. The ecosystem as a metaphor is powerful and therefore prone to misconception and preposterous thinking \citep{OhInnovationExamination}. At the same time, scholars continue to seek the extent in which the biological ecosystem concept can indeed applied. The study on the technospecies concept \citep{Weber2015WhoConcept} is an example of the latter approach. In this dissertation, we are first and foremost interested in the properties that scholars and decision-makers attach to innovation (eco)systems and the ways the investigations of these properties can be conducted with a data-driven visual network analytics approach. In fact, we believe that empirical research can help to take the discussion to more concrete level and provide means to test the validity of innovation ecosystem theory. 

In the experiments included in this dissertation, we investigate innovation ecosystems with a Silicon Valley-style mindset. This means that we do not limit the investigation into companies and their interconnections. Instead, we study the structures that emerge from activities taking place around startups, the individuals that found, advice, invest into, and work in key positions for the startups. Also growth companies that have already crossed the death valley of company growth and continue to evolve toward a liquidity event, e.g. exit through an acquisition or Initial Public Offering (IPO) are of our interest. For context, we also study the already established enterprises and deals and alliances connecting them with each other. Moreover, we seek ways to explore the role of individual customers as well as organizations that facilitate the creation and growth of the companies in the core of our investigations.

\section{Network analysis}
\label{sec:networkanalysis}

Innovation ecosystems are studied in this dissertation with visual network analytics. In network analysis, phenomena under investigation are modeled as nodes and edges representing the entities and their interconnections. We fully subscribe to \cite{Yang2014OverlappingNetworks} in that ``Networks provide a powerful way to study complex systems of interacting objects.'' 
 
Network analysis has its roots in social network analysis (SNA) \citep{Moreno1953,Wasserman1994SocialApplications}. While the first applications of network analysis dates back to 1950's and Milgram's famous experiment that gave evidence on the small-world nature of the world was reported in \citeyear{Milgram1967TheProblem} \citep{Milgram1967TheProblem}, network analysis remains a relatively new method for a number of domains. Key steps in network science include the introduction of a model for generating small-world networks \citep{Watts1998,Watts1999} and the discovery of scale-free networks \citep{Barabasi1999EmergenceNetworks}. The availability of interesting real-life social data and the development of computing capabilities have significantly increased the potential for applying network analysis in investigating various phenomenon \citep[cf.][]{Hansen2011AnalyzingWorld,Bastian2009Gephi:Networks}.

Social network analysis studies the social structures of actors. Sociogram is the core artifact in social network analysis. \cite{Wasserman1994SocialApplications} define sociogram as ``a picture in which people (or more generally, any social units) are represented as points in two-dimensional space, and relationships among pairs of people are represented by lines linking the corresponding points.'' The network structure is key to understanding the complex system of relationships \citep{Barabasi2003Linked:Means}: ``Small changes in the topology, affecting only a few of the nodes or links, can open up hidden doors, allowing new possibilities to emerge.''

A phenomenon under investigation can be modeled either as one-mode, two-mode, or multimodal network. In one-mode networks, all the nodes are of same type. Among company board members, for example, connections between the nodes are formed on basis of friendship or, in innovation ecosystem context, on the basis of company board co-membership. In two-mode networks, there are two types of nodes. A two-mode network of company and investor would show a company node connected to each investment firm from which it has received funding. Means to visualize two-mode networks include hypergraphs and bipartite graphs \citep{Freeman2000VisualizingNetworks, Jesus2009}. The connections may be undirected or directed, the latter resulting into a digraph. Further, the connections between the actors can be either dichotomous or weighted, in which the strength of a connection can be expressed.

The analysis of overall network structure, the different characteristics of the network, the roles of the network actors, and the nuances of their interaction are of interest in many fields of research. The actors in networks, acting as hubs or connectors, may be characterized as random, small-world \citep{Watts1998,Milgram1967TheProblem} or scale-free \citep{Barabasi2003} as they diffuse information within the network \citep[cf.][]{Molka-Danielsen2007IRISAnalysis}. Process phenomena such as preferential attachment \citep{Barabasi1999EmergenceNetworks}, homophily, reciprocity and transitivity \citep[cf.][]{Giuliani2008} shape the networks as they evolve.

Precise SNA metrics can be calculated for all three units analysis in innovation ecosystem investigations: network actors, connections between the actors as well as the network as a whole. Node degree representing the number of connections of a node is the simplest metric. Main categories for actor SNA metrics are centrality and prestige \citep{Wasserman1994SocialApplications}. Key metric for connections is their weight. Metrics such as density and cohesion \citep{Hansen2011AnalyzingWorld} describe networks quantitatively.

Network analysis has been applied to investigate companies and different company-related phenomena. Co-creator networks have been one of the early applications of visual social network analysis. When reviewing the historical and theoretical foundations of SNA, \cite{Freeman2009MethodsVisualization} refers to early work of Hobson who already in 1884 ``produced a visual image of a social network that was not based on kinship.'' Unlike his predecessors, Hobson designed a two-mode network of corporations and their directors with, as \cite{Freeman2009MethodsVisualization} interprets, the rationale ``that, to the degree that corporations shared directors, they could be expected to cooperate and work together.'' Another example of an early visual investigation of corporate networks is Levine's work on ``corporate interlocks'' as relationships through which social norms influence information flow for business intelligence \citep{Levine1979Joint-spaceAlternatives}. The notion of weak ties \citep{Granovetter1973TheTies} is another important landmark in applying network analysis to sociological research in studying business and economy. \cite{Olson2008GreatAPI} reports a more recent example of visualizing corporate interconnections implemented with socially constructed data. \cite{Basole2009VisualizationEcosystem} applies visual network analysis to analyze interfirm relations in the mobile ecosystem and gives an extensive review of literature on visual network analysis in business studies. Importantly, innovation ecosystem studies that utilize network analysis are almost non-existing outside the ones included in this dissertation.

Examples of innovation ecosystem investigations with a network-centric mindset are rare outside the works of the authors and his collaborators. There is, however, an increased interest toward analyzing innovation ecosystems as network. Recent examples of this include \cite{Clarysse2014CreatingEcosystems} and \cite{Parise2015HowIdeas}.

\section{Visual network analytics}

Due to the complexity of business ecosystems, derivation of conceptual insights from business ecosystem data is challenging \citep{Bizzi2012StudyingNetworks,Basole2015UnderstandingApproach}. These challenges exist in innovation ecosystems as well and the openness of the latter further adds to these challenges. Visual revelation of patterns in complex ecosystem data allows for gaining important knowledge of the patterns and dynamics of [innovation] ecosystems \citep{Basole2013UnderstandingVisualization}.

While a statistical analysis provides valuable insights to the structure and dynamics of ecosystems, important knowledge can also be gained through the visual revelation of patterns in a complex business ecosystem data \cite{Basole2013UnderstandingVisualization}. Indeed, visualizations are more than merely artistic approaches to depicting structure in helping investigators to explore the data throughout the analysis process \citep{Fox2011ChangingVisualization}. Visualizations have been used to explore, interpret, and communicate data in order to aid humans in overcoming their cognitive limitations, making structure, patterns, relationships, and themes visible, and providing a means to efficiently compare multiple representations of data in similar fields such as medicine, dentistry, computer science and engineering. \cite{Tufte1983VisualInformation} suggests that when applied properly, visualization is an extremely valuable tool for understanding and analyzing business issues, including strategy, scenario planning, and problem-solving.

Visual analytics is used in this dissertation to negotiate issues related to complexity to gain knowledge on innovation ecosystem ``through the visual revelation of patterns'' \citep{Basole2013UnderstandingVisualization}. We claim that visual analytics allows for understanding and analyzing issues related to business and innovation alike through scenario-planning and problem-solving \citep{Tufte1983VisualInformation}.

We are exited to observe the existence of individual pieces of literature taking note on the development of analytics-based insights that go beyond quantitative (causal) relationships of individual measurements \citep{Bygstad2011InAnalysis,Bendoly2016FitAnalytics,Williams2015MixedAnalysis}. In the context of information systems research, \citep{Bygstad2011InAnalysis} present a critical realist-inspired analytical framework for identifying socio-technical mechanisms in a way that allows for ``ontological depth, creative thinking and more precise explanations'' that goes beyond traditional statistical analysis. \cite{Williams2015MixedAnalysis} remind of the interdisciplinary origins of social network analysis and present a framework for processing data from multiple sources including archival records for network analysis with a mixed methods approach. \cite{Bendoly2016FitAnalytics} put visual analytics into the center of stage in data analytics, from data validation and curation through exploration and discovery to end-result communication and (re-)interpretation. Further, \citeauthor{Bendoly2016FitAnalytics} points to the importance of ``not just simultaneous parallel use [of data and visualizations] but truly joint utilization and team-wise sensemaking for effective decisions.''

With visual network analytics, we refer to taking a visual analytics \citep{Thomas2006AAgenda,Heer2012InteractiveAnalysis} approach to network analysis. Visual analytics stresses the process-centric, interactive nature of using visualizations in supporting data-driven investigations \citep{Keim2010MasteringAnalytics,Heer2012InteractiveAnalysis}. Visual analytics is a particularly suitable approach for exploring new phenomenon with a data-driven approach. Visual network analytics allows the emergence of insights on the structure and dynamics of business and innovation ecosystems \citep{Basole2009VisualizationEcosystem}, social media platforms \citep{Smith2015TheConversations} and other networked phenomena. Transforming those insights into action, however, requires communicating the insights to constituents of change \citep{Russell2011TransformingOrchestration,Still2014InsightsVisualisations}. Visual network analysis is a valuable method for investigating social configurations and for interactively communicating their findings to others \citep{Freeman2000VisualizingNetworks}.

The well-known mantra of visual information retrieval iterates over three phases: ``Overview first, zoom and filter, then details-on-demand'' \citep{Shneiderman1996TheVisualizations}.
In the context of visual network analysis, however, users may prefer to follow the process of ``start with what you know, then grow'' \citep[][p. 35]{Heer2005}. Visual analytics theory suggest that, at best, an investigator of a phenomena is able to interact with all the phases of the process from view creation to exploration and refinement in an expressive manner \citep{Heer2012InteractiveAnalysis}.

A data-driven process for understanding the roles of the different actors in an innovation ecosystem allows for interactive discovery to support both investigation and orchestration of innovation ecosystems \citep{Russell2015RelationalEcosystems}. Multiple perspectives on ecosystem structure and the structural positions of individual actors can be invited and exchanged during the investigative process. With subsequent automation of data updates and tracking analyses, the assumptions and contingencies underlying decisions can be monitored for changes that would impact policy and program directions.

In the experiments included in this dissertation described in detail in Chapter~\ref{ch:experiments}, we use network nodes to represent innovation ecosystem actors. The actors are connected to each other on basis of different kinds of transactions between them. These transactions include investments, acquisitions, deals and alliances. Moreover, key individuals are connected to companies they are or have been affiliated with. Network metrics are used to quantify the structural positions of these actors. Network representations are laid out in a way that allows their visual investigation. Visual properties of network nodes and edges are specified in a way that supports these investigations.

% Very little literature exists on visual (network) analytics of innovation ecosystems. Searching Scopus in April 2016 for (visual  analytics  OR visual  analysis  OR  visualization) AND (innovation ecosystem OR innovation ecosystems) gives only 17 results, including three articles which the author of this dissertation has co-authored.

\section{Cognitive fit of network analysis}

Network representation of information is utilized extensively in information systems. Indeed, network representation of information is at the core of hypertext. When presenting the notion of the hyperlink, Vannevar Bush highlights the similarity in between network representation and the way the human mind operates \citep{Bush1945AsThink}: 

\begin{quote}
``The human mind [...] operates by association. With one item in its grasp, it snaps instantly to the next that is suggested by the association of thoughts, in accordance with some intricate web of trails carried by the cells of the brain. It has other characteristics, of course; trails that are not frequently followed are prone to fade, items are not fully permanent, memory is transitory. Yet the speed of action, the intricacy of trails, the detail of mental pictures, is awe-inspiring beyond all else in nature.''
\end{quote}

We claim that representing innovation ecosystems as networks allows for an intuitive way to investigate and revisit data representing innovation ecosystems. Indeed, \cite{Bush1945AsThink} added that ``[m]an cannot hope fully to duplicate this mental process [the intricacy of trails in human mind] artificially, but he certainly ought to be able to learn from it.'' In this dissertation, we taken advantage of existing sources of digital data on innovation ecosystem actors and transactions and affiliations between them to recreate these traces to allow for system-level views of the currently scattered innovation ecosystem landscape. To reuse McGonigal’s \citeyearpar{McGonigal2005} insightful metaphor inspired by Lewis Carroll, we see visual network analytics as meas to craft ``rabbit holes'' through which innovation ecosystem explorers are drawn into new information landscapes beyond their imagination to find new, interesting ideas, companies, investors, communities or a similar-minded innovation champions \citep[cf.][]{Huhtamaki2007CommunityEcosystem} or emerging patterns that give competitive edge to a company, ecosystem, program, region or nation.

Cognitive fit theory suggests that in supporting problem-solving and decision-making, it is particularly important to find a fit between the problem-solving task and the problem representation and supporting tools; cognitive fit allows for faster and more accurate performance in decision-making \citep{Vessey1991}. There is a delicate balance in developing tools with cognitive fit, however: experimental research shows that neither the time used to conduct a task nor the confidence that the users feel about their decisions are good predictors of the accuracy of their insights or decisions \citep[cf.][]{Dunn2001}. Cognitive fit theory further suggests that when the problem representation fits the problem-solving task, a preferable and more consistent mental representation of the problem will be realized, thereby facilitating the problem solving process, and consequently resulting in preference for the representation, along with faster and more accurate performance in decision-making \citep{Basole2016EnablingAnalysis}.

Using network analysis in the context of investigating and orchestrating innovation ecosystems is closely related to relational and social capital theory. In their seminal article on social capital\footnote{\cite{Still2013relationalsocialcapital} show that \cite{Nahapiet1998} is a key article in between social capital and relational capital research.} \cite{Nahapiet1998} introduce a third dimension  for social capital to complement the relational (relationship-level social capital) and structural dimension (system-level social capital) and label that as cognitive dimension to refer ``to those resources providing shared representations, interpretations, and systems of meaning among parties.'' \cite{Nahapiet1998} ``believe it represents an important set of assets not yet discussed in the mainstream literature on social capital but the significance of which is receiving substantial attention in the strategy domain.''

From relational capital research viewpoint, this dissertation seeks to use visual network analytics to augment the cognitive dimension of ecosystemic relational capital \citep{Still2014EcosystemicRelationalCapital}.