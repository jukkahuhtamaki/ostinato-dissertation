\chapter{Abstract}
\label{chapter:abstract}

% \todo{Reintroduce: what, why, how} 

More often than ever before, innovation activities are crossing organizational boundaries and taking place in between formal organizational structures. This new context for innovation activities is increasingly referred to as innovation ecosystem. Open innovation, co-creation, user driven innovation as well as API and platform economy and business ecosystems are key drivers of the transformation. Innovation ecosystems are open dynamic systems that cross the organizational and geographical boundaries and include financial, technological and political dimensions. Imperatively, talented humans have a focal role in ecosystemic innovation activities. Innovation ecosystems set a new framework for analyzing and investigating and therefore measuring the ecosystems.

Measuring and visualizing innovation is hard. This is particularly the case for innovation ecosystems where innovation activities take very complex forms and even the identification of relevant actors and stakeholders is challenging. At the same time, system-level analysis of innovation ecosystems is imperative for three groups: innovation ecosystem scholars, policy and decision makers, and innovation ecosystem actors. Moreover, new sources of digital data on innovation activities have become available. This dissertation seeks to bridge the gap between opportunities provided by the digital data available and the desire to investigate innovation ecosystems in system level.

In this dissertation, we take an Action Design Research approach to develop means to investigate the structural properties of innovation ecosystems in system level with visual network analytics. We start from the realization that interconnectedness is a key property of innovation ecosystems. Addressing innovation ecosystems as networks, i.e. collections of pairs of interconnected innovation ecosystem actors, allows scholars and practitioners means to gain insight on innovation ecosystem structure and the structural roles of individual ecosystem actors. To experiment how innovation ecosystems should be modeled and analyzed as networks, we investigate a number innovation ecosystems in regional, metropolitan, national and international level as well as e.g. in the context of programmatic activities supporting innovation and growth. The main objective of the dissertation is to develop a process model for visual network analytics of innovation ecosystems in a data-driven manner. 

We show that visual network analytics is a valuable method for investigating and mapping the structure of innovation ecosystems. Transactional microdata on innovation ecosystem actors and their interconnections is collected from various digital sources. Innovation ecosystem actors are represented as network nodes that are connected to each other through transactions including investments, acquisitions as well as advisory, founder, and contributor affiliations. Network metrics are calculated to quantify the structural positions of individual actors. Interactive visual analytics tools are used to support the visual exploration of the innovation ecosystem under investigation with both top-down and bottom-up strategies. 

This dissertation makes several contributions in pushing forward the art and science of data-driven visual network analytics of innovation ecosystems. Most importantly, the dissertation proposes the Ostinato Model, an iterative, user-centric, process-automated model for data-driven visual network analytics. The Ostinato Model simultaneously supports the automation of the process and enables interactive and transparent exploration. The model has two phases, Data Collection and Refinement and Network Creation and Analysis. The Data Collection and Refinement phase is further divided into Entity Index Creation, Web/API Crawling, Scraping, and Data Aggregation. The Network Construction and Analysis phase is composed of Filtering in Entities, Node and Edge Creation, Metrics Calculation, Node and Edge Filtering, Entity Index Refinement, Layout Processing and Visual Properties Configuration. A cycle of exploration and automation characterizes the model and is embedded in each phase.

In addition to the Ostinato Model, this dissertation contributes with a set of design guidelines for modeling and visualizing innovation ecosystems as networks. Finally, we contribute to the empirical body of knowledge on innovation ecosystems through the series of investigations of innovation ecosystems of different levels of abstraction and complexity. Innovation ecosystem scholars, policy makers, orchestrators and other stakeholder of the innovation ecosystem under investigation in this dissertation have subscribed to the presented approach. More research and development of supporting processes and tools is needed to take full advantage of the presented approach in investigating, facilitating, and orchestrating inter-organizational innovation activities.