\chapter{Discussion}
\label{ch:discussion}

The objective of this dissertation is to explore ways network analytics should be applied to investigate the structural properties of innovation ecosystems. Taking a data-driven approach to conduct the investigations was a set to be the key design criteria. That is, it should be possible to collect and aggregate data from various heterogeneous sources in an automated fashion allowing reproducible analysis. More specifically, three individual objectives were set to the dissertation. First, \ref{objective:empirical} was to contribute to the empirical body of knowledge by running a series of investigations on the network structure of innovation ecosystems of different levels of abstraction and complexity. Second, \ref{objective:ecosystemnetworks} was to develop design guidelines on how to model innovation ecosystems as networks for visual analytics. Third and most importantly, \ref{objective:processmodel} was to design a general process model for data-driven visual analytics of innovation ecosystems.

To reach these objectives, an action design research approach was taken to conduct research in two complementing streams. First, a series of investigations on innovation ecosystems was implemented to gain knowledge on the ways investigators as well as innovation ecosystem actors and stakeholders prefer to model the innovation ecosystems as networks. Second, a set of requirements were distilled from the experiments to support the design of a general process model for data-driven visual analytics of innovation ecosystems as networks. Third and most importantly, Ostinato Model was developed through aggregating and extending existing process models in a way that the requirements specific to data-driven visual network analytics of innovation ecosystems are met.

\cite{Gregor2013PositioningImpact} suggest that ``with socio-technical artifacts in IS, when the design is complex in terms of the size of the artifact and the number of components (social and technical), then explicit extraction of design principles'' may be included in the discussion section. Since the dissertation at hand, a compilation of papers accompanied by a summary of the work and key findings, is largely discussion in nature, we have already introduced two key sets of design principles. First, in Chapter~\ref{ch:ecosystemnetworks}, we enumerate a number of design principles for modeling innovation ecosystems as networks for their visual investigation. Second, in Chapter~\ref{ch:ostinato}, we describe the Ostinato Model for data-driven visual investigation of the structure of innovation ecosystems.

In the next sections, we will discuss and review the key contributions of this dissertation work to give evidence that we have bridged the identified research gap. 

\section{Empirical investigation of innovation ecosystems}

The investigations contributing to \ref{objective:empirical} are covered in detail in Chapter~\ref{ch:experiments}. The research in this dissertation was conducted in a multidisciplinary, global team of researchers with background both in business as well as academia. The members of the research team posses expertise and real-life experience in knowledge management, innovation ecosystem orchestration, innovation research, information system design, machine learning, and networks science among other domains. The investigations included in this dissertation explore and describe the innovation ecosystems in a novel way. Through the individual investigations, we have provided new insights into innovation ecosystems of different levels of abstraction and complexity.

Investigations on innovation ecosystems took place in five different levels of abstraction and complexity, in  platform, business domain, innovation program, national, and international level.

On platform level, we joined with Demola operating team to seek ways to use data-driven visual network analytics in representing the structure and dynamics of an ecosystem engager that is targeting to facilitate collaboration between (Tampere-based) universities and companies. In co-creation, we found that animating the evolution of the network structure of Demola platform is a particularly useful approach to present, describe, market, and support selling of the platform for existing and new stakeholders.

On business domain level, we investigated the key pairs of actors in the mobile ecosystem. During the time of the investigation, Nokia and Microsoft had just recently announced a strategic alliance to work together in developing their mobile offering. Google acquired Motorola Mobility in August 2011 to strengthen its capabilities in the mobile domain. The investigation highlights the importance of data triangulation for covering the different aspects of innovation ecosystems. Using Thomson Reuters SDC Platinum, the standard data source in strategy research, Microsoft shows up as the supernode even in network representation of actor network surrounding Google and Motorola Mobility. Only when IEN Dataset is used, Google’s true size accumulated through series of acquisitions and the flow of talented individuals becomes visible.

On program level, we investigated the interconnections in between companies taking part in Tekes Young Innovative Companies program. Two sources of data were used to conduct the study, IEN Dataset and Twitter. We show that connections exist in between the companies that are individually selected into the YIC program. This highlights the importance of ecosystem-level analysis in giving context and therefore supporting decision-making. Moreover, we contribute by using social media data to give a system-level view into those interested in the companies. On a longer term, should the YIC program be successful in selecting and supporting companies in growing, we would see something equal to a food chain of investors and acquirers emerging. Business angels, serial entrepreneurs--either active or successful--also take a key role in such a network representation of the innovation ecosystem around YIC.

On national level, we provided a system-level view into the Finnish Innovation Ecosystem. The system-level view includes already established enterprises, growth companies, and startups as well as investors and key individuals affiliated with the companies. More specifically, we created four different representations of the ecosystem, i.e. microscopic, microscopic, macroscopic, and multiscopic. The results show that a handful of key individuals who entered the global startup ecosystem early and were successful in growing and selling a company have a prominent role in the Finnish Innovation Ecosystem. Nokia is visible through its current role as a source of talent flowing into the ecosystem. Startup Sauna, a student-based initiative for supporting startup creation also has a notable role. Recent success stories Rovio Entertainment and Supercell take a peripheral position; the presented approach does enable monitoring the evolution of the ecosystem around them in the future. Our practical suggestions in national level include active communication and data sharing using a wide variety of media, and utilizing network views for targeted actions as well as for creating shared understanding and vision.

On international level, we took a network orchestration viewpoint in co-creation with the representatives of EIT ICT Labs. Our results indicate that with coordinated and continuously improved use of visual and quantitative social network analysis, special characteristics, significant actors and connections in the innovation ecosystem can be revealed to develop new insights. Creating a network representation of EIT ICT Labs including San Francisco Bay Area as the 7th node is an example of scenario planning that the approach taken in this dissertation enables. We conclude that the Innovation Ecosystem Transformation Framework \citep{Russell2011TransformingOrchestration} is a useful tool for developing shared vision and in supporting the orchestration of innovation ecosystem transformations and show the value of socially constructed data in gaining insights on the structure of a broad-based international innovation ecosystem.

A key part of our contribution to empirical research is the introduction of novel data sources on innovation ecosystems. A number of datasets were used in the investigations, including social media, socially constructed data available online, and proprietary sets of data represented as spreadsheets and other formats. In most of the investigations, we ended up using data sources that were external to the focal organization of the innovation ecosystem under investigation. Only for Demola, the least abstract and least complex of the investigations, we decided to use the project data set that the Demola team is maintaining internally. For the other investigations, we used two main sources of data, i.e. Innovation Ecosystems Network Dataset and Thomson Reuters SDC Platinum. In addition, Twitter data was used in one of the investigations. Moreover, for first investigation of the Finnish Innovation Ecosystem in \ref{pub:finland} we aggregated data from several additional sources. The data sources are covered in detail in Section~\ref{sec:networkdata}.

In all of the investigations included in this dissertation, we have taken advantage of new sources of data from socially constructed to institutionally constructed to data accumulated through the day-to-day operations, something that \cite{Williams2015MixedAnalysis} refer to as archival data. The experiments therefore contribute to organizational research in giving evidence of the usefulness of archival records that are observed as novel in organizational research literature, cf. \cite{Williams2015MixedAnalysis}: ``despite the possible benefits, secondary data are rarely used in organizational social network studies and are almost never considered from a qualitative perspective.'' 

To conclude, we claim that in-house data is particularly useful in supporting an organization in describing their own operations for others in context of marketing and public relations. Data originating from external sources, on the other hand, allows for new viewpoints and novel insights into the structure of innovation ecosystems and the roles of individual actors in them.

\section{Investigating innovation ecosystems as networks}

Innovation ecosystem investigations can take place in three different levels of analysis: actor, relationship, and ecosystem level \citep{Jarvi2016TakingReview}. Most of the investigations on innovation ecosystems focus either on individual firms or pairs of firms and their relationship (dyads). The few ecosystem-level innovation ecosystem investigations focus on the ecosystems run by a single focal company. Recent study of a Flanders-based startup ecosystem \citep{Clarysse2014CreatingEcosystems} presents a notable exception, however the authors claim to investigate a knowledge ecosystem instead of an innovation ecosystem. Moreover, both quantitative and visual analysis included in the paper remains simple.

The investigations included in this dissertation show that representing and analyzing innovation ecosystems as networks adds value to three interrelated fields: the scholarly investigation of innovation ecosystems, innovation ecosystem analytics, and innovation ecosystem orchestration. Guidelines for constructing network representations of innovation ecosystems, analyzing and visualizing the networks, investigating network evolution, the role of interactive exploration and the ability to share the findings provides a basis for consistent analysis of innovation ecosystems in all of these fields. In Chapter~\ref{ch:ecosystemnetworks}, we present a set of design guidelines to support network investigations of innovation ecosystem structure and its evolution and the structural roles of individual actors in the ecosystem. 

Key finding in the investigations is that both innovation ecosystem stakeholders as well as academic co-investigators preferred to model the innovation ecosystems as multimodal networks including key organizational investors and key individuals in addition to firms. For visual analysis, this allows for truly ecosystem-level insights as all the actors are present in individual visualizations. Multimodal networks are, however, not an optimal starting point for quantitative analysis. The key reason for this is the fact that often--in all of the investigations included in this dissertation, in fact--the possible connections between actors are limited: in our investigations, venture capital investors are only connected to companies, not with each other. The same restriction applies for individuals. This means that network-level metrics such as density as well as individual metrics taking into account the larger network structure, e.g. Page Rank, HITS, and eigenvector centrality provide only limited value for analysis compared to a situation where all the actors represented as nodes in the network can be connected with each other through directed nodes. Because of limited connectivity between actors, a network can never be fully connected and therefore density value has to be interpreted with particular care. Similarly, metrics including Page Rank and HITS taking account both the direction of connection as well as treat nodes unevenly--e.g. in Page Rank the authority of a node referring to another node is considered in calculating the value for the reffed node--may be of limited utility in partially connected multimodal networks.

Betweenness centrality was perceived to be a particularly useful metric in the investigations included in this dissertation. Several reasons for its utility exists. To begin, betweenness can be calculated for undirected networks as well as for networks with limited connectivity in between modes of nodes. Moreover, betweenness does take into account the relative position of a node in connecting different parts of the ecosystem. Perhaps most importantly, the principle for calculating betweenness centrality value is relatively straightforward to comprehend. It should be noted, however, that betweenness centrality is prone to errors in the data and therefore investigators should be able to use a set of network metrics for comparison and context as well as to interact with network construction and filtering parameters.

Finally, we would like to point out the steps we have begun taken to more formally validate the utility and usefulness of visual network analytics of innovation ecosystems. \cite{Basole2016EnablingAnalysis} presents the results of an experiment where three different representation of network data is used to support a collection of decision-making task. Moreover, \cite{Russell2015IFKAD} takes more qualitative, descriptive, and reflective stance to describe the ways the network approach contributes to decision-making in the context of innovation ecosystems. 

\section{Main result: Ostinato Model}

\textit{This section is based on \ref{pub:ostinato}.}

The Ostinato Model was developed and validated over multiple investigations serving as experiments that follow Action Design Research. The Ostinato Model has two main phases, Data Collection and Refinement, and Network Creation and Analysis. The Data Collection and Refinement step is divided into Entity Index Creation, Web/API Crawling, Scraping, and Data Aggregation. The Network Creation and Analysis step is composed of Filtering in Entities, Node and Edge Creation, Metrics Calculation, Node and Edge Filtering, Entity Index Refinement, Layout Processing, and Visual Properties Configuration. As a final step, the visualizations are provisioned to investigators and other end users with interactive exploration tools and discussion, and their feedback activates an iteration of the process. A cycle of exploration and automation characterizes the model and is embedded in each phase.

Ostinato Model allows both an exploratory approach during the early phases of the investigation as well as the automation of the data collection and analysis process when the investigative routines grow in maturity. The iteration cycle is especially beneficial in working with multi-source datasets, complex phenomena, and changing externalities that may impact assumptions for decisions, and establishing a dashboard for continued observation of the phenomena, perhaps in real time.

The Ostinato Model has several implications for investigative teams taking the data-driven visual network analytics approach. These will be covered next.

First, facilitation and documentation of the investigative process are required. Low barrier for entry in exploration and analysis poses risks that increase without transparency. Put another way, with added transparency and through intermediate results and easy access, the risk of false conclusions is lowered. Co-ordinated discussion on raw data and its journey to the finalized visualizations and other results is imperative. Documentation of assumptions and rationale for changing data selection or analytical procedures enables transparency. Facilitation also helps in creating literacy of the processes and its outputs within the investigative team. When the intermediate results are available, all the members of the investigative team are able to maintain more of the control of the process and continue to introduce new, novel ways of analyzing the data as their skills and methodological know-how allows.

Second, the cycle of exploration-automation introduces new requirements for the governance of both the data and the analysis process. Intermediary results require transparent authorship in their provenance. The transparent authorship of new datasets, constructed variables, and analytical iterations must be ensured.

Third, starting from exploration and moving toward automation is straightforward with the help of the Ostinato Model. The investigative team is able to move fast in the beginning of the process while, at the same time, maintaining control over the process as its complexity increases. With appropriate technology selections, the process can eventually be relegated to the background to collect, process, analyze and visualize data in an automated manner to support a longitudinal study of a particular innovation ecosystem. And, more importantly, a mature procedure--or one or more of its components-–can be reused to investigate other innovation ecosystem of interest.

Fourth, increased reproducibility is an asset for future investigations but requires explicit governance. Technical reproducibility of the process allows revisiting analytical results of an investigation even after a long time period. Refreshing and collecting new data or, alternatively, adding new dimensions into existing data is straightforward when the process or its individual parts can be run computationally. Rules must be developed for data curation, and access to code and data has to be designed at both the technical and policy levels. 
% Forgot who I wanted to cite here! Got distracted because Mendeley insisted on updating the authentication details due to the integration with Elsevier, cf. https://twitter.com/jnkka/status/721242414606852096 
Governance of the data from raw to intermediate results to outputs as well as the components and software process must be articulated clearly.

Ostinato Model provides blueprints for designing analytical processes with technologies ranging from Python to R and even Javascript. At best, the process is able to support the inclusion of several different technologies, as implemented e.g. by the Wille Visualisation System \citep{Nykanen2008}.

\section{Support for analytics}

After discussing the outcomes of the dissertation through the three key objectives, we will next continue to discuss broader issues related to innovation ecosystem analytics. One important aspect is the support for analytics that the approach derived in this dissertation provides. Both in research as well as in analytics, several different approaches to investigate a phenomena exists. One taxonomy categorizes the approaches into exploratory, diagnostic, descriptive, predictive, prescriptive \citep{Davenport2013Analytics3.0}. 
%\todo{Find a reference for these different categories of research.} Vs. foresight, sensemaking, probabilistic analysis \citep[cf.][]{Silver2012}. 
In the individual investigations, our approach is exploratory and descriptive rather than predictive or prescriptive. Most importantly, ways to facilitate balanced discussion between members of the investigative team with a multidisciplinary members are imperative \citep[cf.][]{Pentland2015,Nunamaker2011TowardSystems}. The data-driven visual network analytics approach presented in this dissertation is, we claim, a major step toward that direction.

Through the transparent creation of visualizations of the innovation ecosystems in different levels of abstraction and complexity from platforms to programs to national and international level, our research does ``map the terrain of a specific phenomenon''\footnote{This nicely formulated phrase originates from an online note ``Research Methods: Some Notes to Orient You,'' see \url{http://isites.harvard.edu/fs/docs/icb.topic851950.files/Research\%20Methods_Some\%20Notes.pdf}} and therefore is taking steps from exploratory to descriptive research sphere. We do realize that with capability to predict, one is able to develop results that truly help in developing new knowledge on how the world works. At the same time, we see that, as an object of research, innovation ecosystems are either complex or chaotic rather than know or knowable. Therefore, we subscribe the argumentation that \cite{Kurtz2003} use to select their approach for developing Cynefin, a sensemaking framework for supporting real-life decision-making and policy-making: 
 
\begin{quote} ``We consider Cynefin a sense-making framework, which means that its value is not so much in logical arguments or empirical verifications as in its effect on the sense-making and decision-making capabilities of those who use it. We have found that it gives decision makers powerful new constructs that they can use to make sense of a wide range of unspecified problems. It also helps people to break out of old ways of thinking and to consider intractable problems in new ways.'' \citep{Kurtz2003}
\end{quote}

In line with the methodological and philosophical foundations of this dissertation described in Section~\ref{sec:methods}, we claim that the Ostinato  Model has a good fit to support innovation ecosystem investigations with a critical realist mindset. Specifically, data-driven visual network analytics gives support to the early phases of the investigative process when research questions eventually leading to the identification of the structure and mechanisms driving a particular phenomenon surfacing as empirical data are only being derived and specified. In addition, the Ostinato Model will in general support the creation, analysis, and validation of network representations created to represent an innovation ecosystem in a transparent and structured manner. 

\section{Artifact evaluation in Action Design Research}
\label{sec:evaluation}

A myriad of approaches exists for evaluating the reliability and validity of research results. In quantitative studies, internal validity, external validity (generalizability), and reliability are the four evaluation criteria. \cite{Shenton2004StrategiesProjects} propose that in qualitative studies, trustworthiness of research can be evaluated through credibility, transferability, dependability and confirmability. In Action Design Research, there are two key axiomatic principle to evaluation: Guided Emergence and Authentic and Concurrentt evalution. \cite{Sein2011ActionResearch} state that ADR emphasizes organizational relevance of the artifact over its technological rigor and the emergence of the artifact through interaction between ADR researcher(s) and the organizational context.

% For compatibility with qualitative research, we will next briefly comment the research in this dissertation through credibility, transferability, dependability and confirmability.

To reiterate, this dissertation seeks to satisfy two key objectives related to Action Design Research:

\begin{itemize}
	\item\ref{objective:ecosystemnetworks} Develop design principles for modeling, representing, and analyzing innovation ecosystems as networks to support their visual investigation.
    \item\ref{objective:processmodel} Develop a process model to support taking a computational approach into the visual investigation of innovation ecosystems in interdisciplinary teams.
\end{itemize}

Next, we will discuss the Building-Intervention-Evaluation cycles or BIE cycles \citep{Sein2011ActionResearch} that have led to the guided emergence of the two key artifacts in this dissertation.

\subsection{BIE cycles for design principles for modeling innovation ecosystems as networks}

The first artifact designed in this dissertation is the set of design principles for modeling innovation ecosystems as networks to support their visual investigation. The design principles are the results of guided emergence that took place in the different investigations serving as experiments on modeling innovation ecosystems as networks. The design principles are described in detail in Chapter~\ref{ch:ecosystemnetworks}. 

Experiments on investigating the Finnish innovation ecosystem mark the starting point of this dissertation research. The investigative team conducted the investigations independently with limited interaction with organizational context, in this case Finnish innovation policy makers at Tekes and Ministry of Employment and Economy. The individual innovation ecosystem visualizations were, however, presented to innovation policy actor through project steering groups and different round table discussions. To explicate, the experiments on the Finnish innovation ecosystem follow IT-dominant BIE \citep[cf.][]{Sein2011ActionResearch}.

Experiment on mapping the innovation ecosystem relevant to EIT ICT Labs was conducted with a Organization-Dominant BIE cycle, i.e. in close interaction with EIT ICT Labs representatives. The premise of the experiment was to use data collected by EIT ICT Labs to represent the innovation ecosystem structure for EIT ICT Labs actors and stakeholders. At the same time, a key objective of the experiment was to investigate the latent structure and existing connections within the EIT ICT Labs co-location cities. After constructing an alpha version of the network representation of EIT ICT Labs using their internal, proprietary data on EIT ICT Labs activities, the investigative team together with EIT ICT Labs representative came to the conclusion that the insights provided by the visualization were already known to EIT ICT Labs actors. 

This realization led to construction of the second alpha prototype of the innovation ecosystem using socially constructed data on companies, their founders, advisors, business angels, and other key individuals as well as organizational investors as the sole data source. This approach turned out to be particularly useful for EIT ICT Labs actors as they were able to observe existing, previously latent connections in between the co-location cities. Moreover, they gained new evidence on the very limited mobility taking place in between the co-locations. Most importantly, however, new insights on the imperative role of venture capital investors, both European as well as US-based in general and Silicon Valley-based in particular, was revealed. Further investigation of Silicon Valley's role led to the design of the most important visualization artifact in this dissertation, namely the representation of EIT ICT Labs ecosystem with Silicon Valley as the hypothetical 7th co-location city of EIT ICT Labs.

The experiment on visualizing the innovation ecosystem of Demola also followed the Organization-Dominant BIE. The author of this dissertation worked in collaboration with Demola operators to design a way to represent the Demola community as a network. This experiment provided a counter-example to EIT ICT Labs with regards to the selection of data source for representing the innovation ecosystem. Guided by Demola operators, we chose to start the network modeling using data on Demola projects and the new connections that project members introduce in between universities and companies that propose the ideas for Demola projects to solve. Moreover, instead of a static visualization, we chose to develop an animation for revealing the evolution and dynamic nature of the Demola plaform in engaging the actors of its extended ecosystem. 

Guided emergence through a series of BIE cycles led to the identification of the key difference between Demola and EIT ICT Labs investigations in the usage of network visualizations. For EIT ICT Labs, the network visualizations were primarily used by the EIT ICT Labs operators in investigating and making sense of the exiting social structures in between the co-location cities. Moreover, insights on Silicon Valley's imperative role led to formulating a new research question: what if Silicon Valley would be the seventh co-location center of EIT ICT Labs? For Demola, the key usage for the developed visualizations and animations was making the Demola process tangible and visible for existing and potential future stakeholders.

\subsection{BIE cycles for Ostinato Model}

The second and most important artifact designed in this dissertation is the Ostinato Model. The model is described in detail in Chapter~\ref{ch:ostinato}. The Ostinato Model is a second-level Generalized Outcome \citep[cf.][]{Sein2011ActionResearch} of this dissertation research, i.e. it something that emerged through conducting several rounds of innovation ecosystem investigations following the design principles described in Chapter~\ref{ch:experiments}.

We will take the next few paragraphs to describe what we mean by the referring to second-level Generalized Outcome.

Following ADR principles, the Ostinato Model originates from a practical need and builds on top of existing theory. To use ADR vocabulary \citep{Sein2011ActionResearch}, the Ostinato Model is equally a result of Practice-Inspired Research as well as an Theory-Ingrained Artifact. The field problem that provided the knowledge-creation opportunity that, through guided emergence, eventually led to the definition of the Ostinato Model stems from the individual experiments where we engaged with innovation ecosystem scholars (i.e. researchers), innovation ecosystem operators (i.e. practitioners) and--towards the end  of the dissertation research process and beyond--with actors and stakeholders of the investigated innovation ecosystems (i.e. end-users).

During the experiments, we observed the existence of several repeating phases in implementing the data processing functionalities to support the innovation ecosystem investigations. At the same time, however, the experimentation-specific requirements often insisted on tailoring these phases in a major way from one experiment to another. This variation suggested us that implementing a general-purpose software was not possible in short term.

Therefore, instead of building a software, we dug deep into existing literature and theory of process models related to data-driven visual investigations covered in detail in Chapter~\ref{ch:processmodels}. This brings in the ADR principle of the Theory-Ingrained Artifact \citep{Sein2011ActionResearch}. Key leads toward process model literature were the Network Analysis and Visualization (NAV) process model \citep{Hansen2012DoData} and Derek Hansen's talk on infrastructure for supporting computational social science \citep{Hansen2013InfrastructureScience}. The final push that inspired us to externalize our accumulated knowledge on the data-driven visual analytics process of innovation ecosystems came from the Kredible.net community\footnote{Kredible.net: Understanding roles and authority in knowledge markets – An NSF Funded Project. Award No. 1244708, \url{http://kredible.net/in/}}. We participated in a Kredible.net workshop on Reputation, Trust and Authority Workshop at Stanford University\footnote{Kredible.net workshop at Stanford on October 2013, \url{http://kredible.net/in/second-kredible-net-workshop-stanford-university/}} to present our ideas on infrastructure for data-driven visual network analytics of innovation ecosystems. The Kredible.net community took up the idea and accepted our proposal for a book chapter on the topic \citep{Huhtamaki2015Ostinato:Analytics}.

The final version of the Ostinato Model is a result of a number of iteration rounds in which individual investigations and experiments were analyzed for processual steps and their interconnections. The validity of the Ostinato Model is further evaluated and described in \cite{Huhtamaki2017ProcessingExperiences} where a number of the investigations included in this dissertation are analyzed through the Ostinato Model lense.

% \section{Evaluating results through qualitative research criteria}

% To be compatible with evaluation criteria more general to those included in core Action Design Research, we will next briefly comment the results through evaluation criteria applied in the context of qualitative research. In quantitative research, the canonical evaluation criteria include construct validity, internal validity, external validity, and reliability (ensuring that someone else can arrive to the same results by repeating the same case study). \cite{Yin2003CaseMethods} proposes these four general principles for evaluating the validity and reliability of case study-based social science studies.

% The proposed qualitative research counterparts to quantitative research validity and reliability evaluation criteria include \citep{Jussila2015SocialInnovation} are transferability: ``To evaluate the trustworthiness of qualitative studies, four constructs corresponding to the criteria employed by positivist investigators are proposed: 1) credibility (in preference to internal validity); 2) transferability (in preference to external validity/generalizability); 3) dependability (in preference to reliability); and 4) confirmability (in preference to objectivity) (Guba, 1981; Denzin \& Lincoln, 2000; Shenton, 2004).''

% In the series of experiments conducted in this dissertation, particularly in \ref{pub:demola} and \ref{pub:eitictlabs}, we communicate and collaborate actively with the innovation ecosystem orchestrators, operators and stakeholders. These actors are knowledgeable and have a holistic understanding of the context, thus using them as informants is seen applicable, particularly because they are interested and open to collaboration. 

% \section{Evaluation: Design Science guidelines}
% \todo{These probably need to go.}

% To make sure that we have conducted the investigations in this dissertation in a rigorous manner in line with Design Science Research principles, we will next reflect dissertation results through Design Science guidelines presented in Section~\ref{sec:methods}. To ensure the tractability of the discussion, we will focus the discussion to Ostinato Model, the key result and artifact produced in this dissertation.

% \ref{guideline:artifact} Design as an Artifact. Ostinato Model is the artifact produced in this dissertation. Moreover, in individual experiments, we have created a collection of network models of innovation ecosystems of different levels of abstraction and complexity. The viability of the Ostinato Model was demonstrated through its use in the experiments. 

% \ref{guideline:relevance} Problem Relevance. The Ostinato Model allows multidisciplinary teams to investigate innovation ecosystems with a data-driven approach. Innovation ecosystems are of growing importance to regional, national, and international innovation actors and their stakeholders and therefore the Ostinato Model contributes to a problem space of high relevance.

% \ref{guideline:evaluation} Design Evaluation. The Ostinato Model was developed over a series of experiments through which its usefulness, viability, and fit to support exploration of innovation system structure with a data-driven visual analytics approach has been evaluated. The experiments were conducted in multidisciplinary teams and in collaboration with innovation ecosystem stakeholders. The Ostinato Model as well as the individual innovation ecosystem network models have been presented both to scholars in conferences as well as to innovation ecosystem actors and stakeholders in workshops and seminars.

% \ref{guideline:contributions} Research Contributions. The Ostinato Model is a novel contribution to the field of computational social science in general and specifically to the emerging field of innovation ecosystem research. Further, the generalization of the model has already been tested in the context of research on communities operating on Twitter.

% \ref{guideline:rigor} Research Rigor. Despite the exploratory nature of the research process, we have made explicit efforts to ensure research rigor. More specifically, we have ... \todo{Learn how to argument for ensuring research rigor.} 

% \ref{guideline:search} Design as a Search Process. The development of the Ostinato Model has been conducted by a multidisciplinary team of innovation ecosystem scholars and in collaboration with "problem environment" actors and stakeholders. The search process has been extended over a series of experiments, five of which are included in this dissertation.

% \ref{guideline:communication} Communication of Research. In addition to publishing the articles describing the results of individual experiments and presenting a number of them in conferences, the authors of this dissertation has instructed the use of the Ostinato model to several business ecosystem and business-oriented Computational Social Science scholars. Moreover, the individual innovation ecosystem network models have been presented in a several seminars and workshops targeted to innovation ecosystem actors, policy makers and other stakeholders.