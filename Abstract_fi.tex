\chapter{Tiivistelmä}

Innovaatiotoiminta ylittää enenevässä määrin organisaatioiden rajat ja siirtyy organisaatioiden ulkopuolelle ja väliin. Tähän innovaatiotoiminnan uuteen kontekstiin viitataan innovaatioekosysteemin käsitteellä. Avoin innovaatio, yhteiskehittely, käyttäjälähtöisyys, API- ja alustatalous ja liiketoimintaekosysteemit ovat muutoksen keskeisiä ajureita. Innovaatioekosysteemit ovat kokonaisvaltaisia avoimia dynaamisia järjestelmiä, joilla on sekä poliittinen, taloudellinen että teknologinen ulottuvuus. Lahjakkailla yksilöillä on aivan keskeinen rooli ekosysteemisessä innovaatiotoiminnassa. Liiketoimintaekosysteemeistä kumpuava innovaatioekosysteemien teoria määrittelee uuden viitekehyksen innovaatiotoiminnan analyysille ja tutkimukselle ja siten innovaatiotoiminnan mittaamiselle. 

Innovaatiotoiminnan mittaaminen ja visualisointi on haastavaa. Innovaatioekosysteemeissä haasteet lisääntyvät entisestään innovaatiotoiminnan kompleksisuuden takia. Jopa olennaisten toimijoiden ja sidosryhmien tunnistaminen on vaikeaa. Samalla on niin, että innovaatioekosysteemien järjestelmätason analyysi on avainasemassa kolmelle ryhmälle: innovaatioekosysteemien tutkijoille, politiikan ja innovaatiotoiminnan päätöksentekijöille sekä innovaatioekosysteemien toimijoille. Innovaatiotoimintaa edustavaa digitaalista dataa on saatavilla aikaisempaa enemmän ja periaatteelliset mahdollisuudet järjestelmätason analyysiin ovat siten parantuneet. Tässä väitöskirjatyössä tavoitteena onkin edistää mahdollisuuksia digitaalisen datan soveltamiseen innovaatioekosystemien järjestelmätason analyysin toteuttamisessa. 

Tässä toimintadesigntutkimuksen otteella toteutetussa väitöskirjassa kehitetään tapoja tarkastella innovaatioekosysteemien rakenteellisia ominaisuuksia järjestelmätasolla visuaalisen verkostoanalytiikan keinoin. Keskeinen lähtökohta tutkimustyölle on havainto siitä, että  toimijoiden väliset kytkökset ovat avainasemassa innovaatioekosysteemeissä. Verkostoanalyysi tarjoaa luontevan keinon innovaatioekosysteemien rakenteellisen analyysin toteuttamiseen. Verkostoanalyysi antaa innovaatioekosysteemien tutkijoille ja toimijoille mahdollisuuden tehdä havaintoja innovaatioekosysteemien rakenteesta ja toimijoiden rakenteellisista rooleista. Tutkimme tässä työssä joukon erilaisia innovaatioekosysteemejä alustapohjaisesta kansalliseen ja kansainväliseen sekä kasvua tukevaan ohjelmatoimintaan tavoitteenamme tunnistaa tapoja innovaatioekosysteemien mallintamiseen ja analysointiin verkostoina. Työn päätavoite on kehittää prosessimalli innovaatioekosysteemien rakenteelliseen tarkasteluun visuaalisen analytiikan keinoin  datalähtöisen laskennallisen analytiikan tuella. 

Osoitamme tässä työssä että verkostoanalyysi tuo lisäarvoa innovaatioekosysteemien verkostorakenteen kartoittamiseen ja tutkimiseen. Innovaatioekosysteemien toimijoita ja heidän vuorovaikutustaan edustavaa transaktionaalista mikrodataa kerätään työssä moninaisista digitaalisista lähteistä. Innovaatioekosysteemin toimijat esitetään verkoston solmuina, jotka kytketään toisiinsa transaktioiden ja muiden yhteyksien perusteella. Sijoitukset, yrityshankinnat ja erilaiset sopimukset sekä neuvonantajana, perustajana tai keskeisenä työntekijänä toimiminen ovat esimerkkejä yhteyksistä. Verkostoanalyysin tunnusluvut mahdollistavat erilaisten toimija- ja järjestelmätason ominaisuuksien esittämisen lukuarvoina. Verkostot visualisoidaan eksploratiivisen analyysin ja tulosten raportoimisen tueksi vuorovaikutteisilla välineillä, jotka mahdollistavat joustavan liikkumisen sekä yksityiskohtia että kokonaisuuksia valottavien näkymien välillä. 

Tämä väitöskirjatutkimus edistää innovaatioekosysteemien datalähtöisen visuaalisen verkostoanalyysin teoriaa ja käytäntöä useilla tavoilla. Väitöskirjatutkimuksen keskeinen tulos on Ostinato-malli, joka mahdollistaa iteratiivisen, käyttäjäkeskeisen tavan toteuttaa automatisoituja datalähtöisen visuaalisen verkostoanalyysin prosesseja. Ostinato-malli jakaa innovaatioekosysteemin analyysiprosessissa kahteen vaiheeseen, 1) datan kerääminen ja jalostaminen sekä 2) verkoston luominen ja analyysi. Datan kerääminen ja jalostaminen toteutetaan neljässä vaiheessa: entiteetti-indeksin luominen, webin ja ohjelmointirajapintojen ryömiminen, datan raapiminen, ja datan koostaminen. Verkoston luominen ja analyysi muodostuu seitsemästä vaiheesta: entiteettien valinta, solmujen ja yhteyksien luominen, tunnuslukujen laskenta, solmujen ja yhteyksien suodattaminen, entiteetti-indeksi täsmentäminen ja visuaalisten ominaisuuksien määrittely. Vuorottelu eksploraation ja automatisoinnin välillä leikkaa läpi prosessin vaiheiden. 

Ostinato-mallin ohella tässä väitöskirjassa määritellään joukko ohjenuoria innovaatioekosysteemien verkostomallinnuksen ja visuaalisen verkostoanalyysityön tueksi. Väitöskirja antaa myös panoksensa innovaatioekosysteemien empiiriseen tietämykseen mallintamalla joukon eri abstraktio- ja kompleksisuustasoja edustavia innovaatioekosysteemejä. Tutkimuksen kohteena olleiden innovaatioekosysteemien tutkijat, politiikan päätöksentekijät, orkestroijat ja muut toimijat ovat omaksuneet esitetyn lähestymistavan. Lähestymistavan kattava hyödyntäminen organisaatiorajat ylittävien innovaatiotoimien tutkimisessa, tukemisessa ja orkestroinnissa edellyttää lisää tutkimusta ja tuotekehitystä.